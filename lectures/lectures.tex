\documentclass{scrartcl}
\usepackage[utf8]{inputenc}
\usepackage[T2A]{fontenc}
\usepackage[russian]{babel}
\usepackage{amssymb}
\usepackage{amsmath}
\usepackage{amsthm}
\usepackage{listings}
\newtheorem{theorem}{Теорема}
\newtheorem{definition}{Определение}
\newtheorem{corollary}{Следствие}[theorem]
\newtheorem{lemma}[theorem]{Лемма}
\title{Лекции по математическому анализу}
\author{Титилин Александр}
\date{}
\begin{document}
\maketitle
\section{Равномерная непрерывность.}
\[
	\forall x \in D \forall  \varepsilon > 0 \exists  \delta \forall y |x - y| < \delta\implies |f(x) -f(y)| < \varepsilon
	.\]
Определение непрерывной функции.
\[
	\forall  \varepsilon > 0 \exists  \delta >0 \forall x \forall y |x- y| < \delta \implies |f(x) - f(y)| < \varepsilon
	.\]
Определение равномерно непрерывной функции.
\subsection{Примеры}
\begin{enumerate}
	\item $f(x) =  C$
	      Берем любую дельту
	      \[
		      |f(x) - f(y)| = 0 < \varepsilon
		      .\]
	\item $f(x) = ax + b$
	      \[
		      |f(x) - f(y)| = |ax + b - ay -b| =|a(x - y)| = |a||x - y|
		      .\]
	      \begin{enumerate}
		      \item $a = 0 \implies$ прошлый пункт
		      \item $a \neq 0$
		            \[
			            \delta = \frac{\varepsilon}{|a|}
			            .\]
	      \end{enumerate}
	\item $f(x) =  x^2 D = [a,b] a\neq b$
	      \[
		      \varepsilon > 0
		      .\]
	      \[
		      |x^2 - y^2| = |(x - y)||(x + y)| \le  (|x| + |y|)|x - y| \le  2C|x - y|
		      .\]
	      \[
		      C = \max(|a| , |b|)
		      .\]
	      \[
		      \delta = \frac{\varepsilon}{2C}
		      .\]
	\item $f(x) = \frac{x^2}{x^2 + 1}$
	      \[
		      | \frac{x^2}{x^2 + 1} - \frac{y^2}{y^2 + 1} | =
		      |\frac{x^2y^2 + x^2 - x^2y^2 - y^2}{(x^2 + 1)(y^2 + 1)}| =
		      |\frac{x^2 - y^2}{(x^2 + 1)(y^2 + 1)}| =
		      |\frac{(x + y)(x - y)}{(x^2 + 1)(y^2 + 1)}|
		      .\]
\end{enumerate}
\begin{definition}
	Пусть $f: D \to \mathbb{R}$ говорят, что функция $f$ удовлетворяет условию Липшица
	если
	\[
		\exists  C \ge  0 \forall  x, y \in D : |f(x) - f(y)| < C|x - y|
		.\]
\end{definition}
\begin{theorem}
	Если функция удовлетворяет условию Липщица на промежутке D, то она равномерно
	непрерывна на промежутке D.
\end{theorem}
\begin{theorem}
	Пусть функция задана на промежутке, функция удовлетворятет условию Гёльде,
	если
	\[
		\exists  C \ge  0 \exists  \alpha > 0 \forall  x, y \in D : |f(x)-f(y)|\le  C|x - y|^{\alpha}
		.\]
\end{theorem}
\begin{theorem}
	Если функция удовлетворяет условию Гёльдера, то она равномерно непрерывна.
\end{theorem}
\begin{proof}
	\[
		C > 0 ~  |x - y| < (\frac{\varepsilon}{C})^{\frac{1}{\alpha}}
		.\]
	\[
		\delta = (\frac{\varepsilon}{C})^{\frac{1}{\alpha}}
		.\]
\end{proof}
\subsection{Пример.}
\[
	D = (0,+\infty) , f(x) = \sqrt{x}
	.\]
\[
	|f(x) - f(y)| = |\sqrt{x}  - \sqrt{y} |
	.\]
\begin{theorem}
	Функция f равномерно непрерывна $\iff$ $\forall  (x_{n}), (y_{n}), x_{n}, y_{n} \in D
		\implies f(x_{n}) - f(y_{n}) \to 0$
\end{theorem}
\begin{proof}
	\begin{enumerate}
		\item
		      Возьмем $\forall (x_{n}) (y_{n}) x_{n} - y_{n} \to 0$.
		      Нужно доказать $\forall \varepsilon >0 \exists  n_0 \forall  n\ge n_0
			      |f(x_{n}) - f(y_{n})| < \varepsilon$
		      Дальше смотрим определение равномерной непрырывности и все понятно.
		\item
		      Теперь обратно доказываем. От противного
		      \[
			      \exists  \varepsilon > 0 \forall \delta > 0 |x - y| < \delta \land |f(x) - f(y)| > \varepsilon
			      .\]
		      Берем все такие последовательность $(x_{n}) (y_{n}) x_{n} - y_{n} < \frac{1}{n}$
		      \[
			      |x_{n} - y_{n}| \to 0 \implies f(x_{n}) - f(y_{n}) \to 0
			      .\]
		      По т. о ментах.
	\end{enumerate}
\end{proof}
\subsection{Пример функции, которая не является равномерно непрерывной}
\[
	D = (0,+\infty) f(x) = \frac{1}{x}
	.\]
\[
	x_{n} = \frac{1}{n}
	.\]
\[
	y_{n} = \frac{1}{n + 1}
	.\]
\[
	x_{n} - y_{n}= \frac{1}{n(n + 1)} \to 0
	.\]
\[
	f(x_{n}) = n
	.\]
\[
	f(y_{n}) = n + 1
	.\]
\[
	f(x_{n}) - f(y_{n}) = -1 \not \to 0
	.\]
\subsection{Еще такой пример}
\[
	f(x) = \ln{x} ~ D = (0;+\infty)
	.\]
\[
	x_{n} = \frac{1}{n}
	.\]
\[
	y_{n} = \frac{1}{n^2}
	.\]
\[
	x_{n} - y_{n} \to 0
	.\]
\[
	f(x_{n}) - f(y_{n}) = \ln{x_{n}} - \ln{y_{n}} = \ln{\frac{x_{n}}{y_{n}}} = \ln{0} \to +\infty
	.\]
\subsection{}
\[
	f(x) = x^2, D =\mathbb{R}
	.\]
\[
	x_{n} =  n + \frac{1}{n}
	.\]
\[
	y_{n} = n
	.\]
\[
	x_{n} - y_{n} \to 0
	.\]
\[
	f(x_{n}) = n^2 + 2 + \frac{1}{n^2}
	.\]
\[
	f(y_{n}) =n ^2
	.\]
\[
	f(x_{n}) - f(y_{2}) = 2 + \frac{1}{n} \to 2
	.\]
\subsection{}
\[
	f(x) = \tg{x} D = (-\frac{\pi}{2};\frac{\pi}{2})
	.\]
\[
	x_{n} = \frac{\pi}{2} - \frac{1}{n}
	.\]
\[
	y_{n} = \frac{\pi}{2} - \frac{1}{2n}
	.\]
\[
	\tg{x_{n}} - \tg{y_{n}} \not \to 0
	.\]
\subsection{Пример}
\[
	D = [0,\frac{1}{2}]
	.\]
\[
	f(x) =
	\begin{cases}
		\frac{1}{\ln{x}}, \text{если } x \neq 0 \\
		0, \text{если } x = 0
	\end{cases}
	.\]
Докажем, что не удовлетворяет условию Гельдера.\\
Пусть $\exists  C >0 \alpha >0 $
\[
	\forall  x, y \in [0; \frac{1}{2}] ~ |f(x) - f(y)| \le  C |x - y|^{\alpha}
	.\]
\section{Теорема Кантора}
\begin{theorem}[Кантора]
	Пусть функция f задана на замкнутом и ограниченном промежутке и непрерывна.Тогда $f$ равномерна непрерывна.
\end{theorem}
\begin{proof}
	Пусть $f:[a,b] \to \mathbb{R}$ непрерывна на $[a,b]$, но не является равномерно непрерывной.
	Тогда $\exists  \varepsilon \forall  \delta >0$
	Найдутся такие $x,y \in[a,b]$, такие что  $|x - y| < \delta \land |f(x) - f(y)| \ge \varepsilon$.
	В частности $\forall  n \in \mathbb{N}$ найдутся такие $x_{n},y_{n}$,
	что $|x_{n} - y_{n}| < \frac{1}{n} \land |f(x_{n}) - f(y_{n}) \ge  \varepsilon$
	Рассматриваем последовательность $(x_{n})$ Выбрали из нее сходящую подпоследовательность $(x_{n_{k}}) \to c c \in [a;b]$
	Рассмотрим $y_{n_{k}}$
	\[
		y_{n_{k}} = (y_{n_{k}} - x_{n_{k}}) + x_{n_{k}}
		.\]
	По ментам
	\[
		0 \le  |y_{n_{k}} - x_{n_{k}}| <\frac{1}{n} \to 0
		.\]
	f непрерывна в точке с $x_{n_{k}} \to c \implies f(x_{n_{k}}) \to f(c)$
	\[
		y_{n_{k}} \to c \implies  f(y_{n_{k}}) \to f(c)
		.\]
	\[
		f(x_{n_{k}}) - f(y_{n_{k}}) \to 0
		.\]
	Противоречие так как $|f(x_{n})  - f(y_{n})| \ge  \varepsilon$
\end{proof}
\begin{theorem}
	\[
		d : (a,b) \to \mathbb{R}
		.\]
	Пусть f непрерывна  f равномерно непрерывна $\iff$
	Существуют конечные пределы   $\lim_{x \to a^{+}} f(x) ,
		\lim_{x \to b^{-}}f(x) $
\end{theorem}
\begin{proof}
	В обратную сторону. Пусть
	\[
		\lim_{x \to a^{+}}  f(x) = A
		.\]
	\[
		\lim_{x \to b^{-}}  f(x) = B
		.\]
	\[
		\overline{f} [a,b] \to \mathbb{R}
		.\]
	\[
		\overline{f(x)} =
		\begin{cases}
			f(x), x \in (a,b) \\
			A , x = a         \\
			B, x = b          \\
		\end{cases}
		.\]
	$\overline{f}$ непрерывна в $x \in (a,b)$ в  $a,b$
\end{proof}
\subsection{Примеры}
$f(x) = \sqrt{x} ~ D = (0,1) $ Так же она равномерно непрерывна на $[a,b]$
\[
	\lim_{x \to 0^{+}} f(x) = 0
	.\]
\[
	\lim_{x \to 1^{-}} f(x) = 1
	.\]
\[
	f(x) = \tg{x} ~ D = (-\frac{\pi}{2}; \frac{\pi}{2})
	.\]
\[
	\lim_{x \to \frac{\pi}{2}^{+} }f(x)  = +\infty
	.\]
f не является равномерной непрерывной\\
\[
	f(x) = \sqrt{1 - x^2}  ~D = [-1;1]
	.\]
По т Кантора равномерно непрерывна.
\section{Интеграл Римана}
\subsection{Разбиение отрезков}
\begin{definition}[Разбиение]
    $[a,b]$ разбиение отрезка на n частей это набор точек  $\tau = x_1,x_2,\dots,x_{n - 1}$
    \[
    a = x_{0}
    .\] 
    \[
    b =  x_{n}
    .\] 
    Номер промежутка -- номер правого конца.
\end{definition}
\begin{definition}[продолжение, измельчение]
Пусть $\tau,\tau'$ $\tau'$ будем называть продолжением разбиения  $\tau$, если 
\[
\tau \subset \tau'
.\] 
\end{definition}
\begin{theorem}
    Любые два разбиения $\tau, \tau'$ отрезка $[a,b]$ имеют общеее измельчение
\end{theorem}
\begin{proof}
    Общее измельчение $\tau \cup \tau'$
\end{proof}
\begin{definition}
    Длина i-того промежутка $\Delta x_{i} = x_{i} - x_{i - 1}$
\end{definition}
\begin{definition}[Разбиение с отмеченными точками.]
    В каждом промежутке разбиения $\tau$ выбрано по точке  $\xi_{i} \in [x_{i - 1},x_{{i}}]$ 
    \[
    \xi = (\xi_1,dots,\xi_{n})
    .\] 
    $(\tau,\xi)$ разбиение с отмеченными точками
\end{definition}
\begin{definition}[Ранг разбиения]
    \[
        \lambda(\tau) = \max{\Delta x_1,\dots, \Delta x_{n}}
    .\] 
\end{definition}
\subsection{Интегральные суммы}
\begin{definition}[Интегральные суммы]
    \[
        f : [a,b] \to \mathbb{R}
    .\] 
    Пусть $(\tau,\xi)$ -- оснащенное разбиение
    \[
    S(f,( \tau,\xi )) = \sum_{i = 1}^{n} f(\xi_{i}) \Delta x_{i}
    .\] 
\end{definition}
\subsubsection{Свойства интегральных сумм}
\begin{enumerate}
    \item $\alpha \in \mathbb{R}~ S(\alpha f,(\tau,\xi)) = \alpha S (f,(\tau,\xi))$ 
        \begin{proof}
            \[
            S(\alpha f,(\tau,\xi)) = \sum_{i = 1}^{n} \alpha f(\xi_{i}) \Delta x_{i} = 
            \alpha \sum_{i = 1}^{n} f(\xi_{i})\Delta x_{i}
            .\] 
        \end{proof}
    \item 
        \[
        S(f + g,(\tau,\xi)) = S(f,(\tau,\xi)) + S(g,(\tau,\xi))
        .\] 
        \begin{proof}
            \[
        \sum_{i = 1}^{n} (f + g) (\xi_{i}) \Delta x_{i} =
        \sum_{i = 1}^{n} (f(\xi_{i}) + g(\xi_{i})) \Delta x_{i}=
        \sum_{i = 1}^{n} (f(\xi_{i}) \Delta x_{i} + g(\xi_{i}) \Delta x_{i}) 
            .\] 
            \[
            \sum_{i = 1}^{n} f(\xi_{i}) \Delta x_{i} +
            \sum_{i = 1} ^{n} g(\xi_{i}) \Delta x_{i}
            .\] 
        \end{proof}
\item 
    Если $f \le  g$, то
    \[
    S(f,(\tau,\xi)) \le  S(g,(\tau,\xi))
    .\] 
    \[
    \forall i f(\xi_{i}) \le g(\xi_{i})
    .\] 
    \[
    f(\xi_{i}) \Delta x_{i} \le g(\xi_{i}) \Delta x_{i}
    .\] 
\end{enumerate}
\subsubsection{Предел интегральных сумм}
Пусть  для любой последовательности $\lambda(\tau^{(n)}) \to 0 ~ n \to \infty$
\subsubsection{Примеры}
\begin{enumerate}
    \item $f(x) =  C, x \in [a,b]$
         \[
        S(f,(\tau,\xi)) = \sum_{i}^{c} C \Delta x_{2} = C(b - a)
        .\] 
    \item 
        \[
            f(x) = 
            \begin{cases}
                1 , x \in \mathbb{Q}\\
                0, x \notin \mathbb{Q}
            \end{cases}
        .\] 
        Разбили на n равных частей. $\lambda(\tau^{(n)}) = \frac{1}{n} \to 0 ~ n  \to \infty$
    \item C -- число $\neq 0$
        \[
            f:[a,b] \to \mathbb{R}
        .\] 
        \[
            c \in [a,b]
        .\] 
        \[
        f(x) = 
        \begin{cases}
            0, x \neq c\\
            C, x = c
        \end{cases}
        .\] 
\end{enumerate}
\begin{theorem}
    \[
    I = \int_{a}^{b} f \iff
    \forall \varepsilon > 0 ~ \exists  \delta > 0 ~ \forall (\tau,\xi)
    \lambda(\tau,\xi) < \delta \implies |S(f,(\tau,\xi)) - I| < \varepsilon
    .\] 
    Будем говорить, что некоторое свойство выполняется для всех достаточно мелких разбиений, если $\exists  \delta >0 $ что это свойство выполняется для всех оснащенных разбиений\\
    $\forall  \varepsilon |S(f,(\tau,\xi)| < \varepsilon$ для достаточно мелких разбиений.
\end{theorem}
\begin{theorem}
    Если функция интегрируема, то она ограничена.
\end{theorem}
\begin{proof}
    Докажем, что f ограничена сверху от противного. Пусть $\forall  C \exists  x, 
    f(x) > C$. Построим последовательность оснащенных разбиений $(\tau^{(n)},\xi^{(n)})$ такую что $\lambda(\tau^{(n)},\xi^{(n)}) \to 0 ~n \to \infty$ 
    \[
    S(f,(\tau^{(n)},\xi^{(n)})) \to \infty
    .\] 
    Так как f не ограничена сверху, то  существует промежуток разбиения, где есть точка $\xi_{i}, f(\xi_{i}) > C$. На всех промежутках кроме этого выберем точку.
\end{proof}
\begin{theorem}
    Пусть функция f задана на отрезке $[a,b]$. И пусть  $K$ -- конечное подмножество отрезка  $(a,b)$ $\Xi$ -- множество всех оснащенных разбиений $(\tau, \xi)$ отрезка, 
    таких что  $K \subset \tau$.
    Тогда для любой последовательность оснащенных размещений из X, такие что длина разбиения стремится к нулю. Тогда предел интегральных сумм -- это интеграл.
\end{theorem}
\subsection{Простейшее свойство интеграла}
\begin{enumerate}
    \item $f \in R_{[a,b]} , \alpha \in \mathbb{R} \implies \alpha f \in R_{[a,b]} \land \int\limits_{a}^{b} \alpha f = \alpha \int\limits_{a}^{b} f 
    $
    \begin{proof}
        Возьмем $\forall (\tau^{(n)}, \xi^{(n)}) \lambda(\tau^{(n)})\to 0$ 
        \[
        S(\alpha f; (\tau^{(n)}, \xi^{(n)})) = \alpha S(f;(\tau^{(n)});\xi^{(n)}))
        .\] 
        \[
        S(f;(\tau^{(n)});\xi^{(n)})) \to \int\limits_{a}^{b} f 
        .\] 
        \[
        \alpha S(f;(\tau^{(n)});\xi^{(n)})) \to \alpha \int\limits_{a}^{b} f 
        .\] 
    \end{proof}
\item
    $f,g \in R_{[a,b]} \to f + g \in R_{[a,b]} \land \int\limits_{a}^{b} (f + g) =
    \int\limits_{a}^{b} f +  \int\limits_{a}^{b} g 
    $
    \begin{proof}
        Рассмотрим $\forall (\tau^{(n)},\xi^{(n)})$  $\lambda(\tau^{(n)}) \to 0$ 
        \[ 
            S(f + g; (\tau^{(n)},\xi^{(n)})) = S(f;(\tau^{(n)},\xi^{(n)})) +
S(g;(\tau^{(n)},\xi^{(n)})) \to \int\limits_{a}^{b} f +  \int\limits_{a}^{b} g 
        .\] 
    \end{proof}
\item 
    $f, g \in R_{[a,b]} \land \forall  x \in [a,b] f(x) \le g(x)$, то
    $\int\limits_{a}^{b} f \le  \int\limits_{a}^{b} g 
    $
\begin{corollary}
    Пусть $f \in R_{[a,b]}$ и пусть числа $m,M$ таковы, что  $\forall  x \in [a;b] 
    m \le  f(x) \le  M$ тогда 
    \[
    m(b-a) \le  \int\limits_{a}^{b} f \le  M(b - a) 
    .\] 
\end{corollary}
\begin{proof}
    \[
    m \le  f \le  M
    .\] 
    \[
    \int\limits_{a}^{b} m \le  \int\limits_{a}^{b} f \le  \int\limits_{a}^{b}  M 
    .\] 
    \[
    m(b - a) \le  \int\limits_{a}^{b} f \le  M(b - a) 
    .\] 
\end{proof}
\item $f \in R_{[a,b]}, g : [a,b] \to \mathbb{R}$ и отличается от f в конечном числе точек. Тогда $g \in R_{[a,b]}$ и $\int\limits_{a}^{b} f = \int\limits_{a}^{b} g 
$ 
\begin{proof}
    Рассмотрим функцию h задананную на $[a,b]$
     \[
    h(x) =  g(x) - f(x)
    .\] 
    \[
    \int\limits_{a}^{b} g = \int\limits_{a}^{b} f + \int\limits_{a}^{b} h = \int\limits_{a}^{b} f  
    .\] 
\end{proof}
\item 
    \begin{theorem}
        $f:[a,b] \to \mathbb{R} , c\in (a;b)$
        \[
            f \in R_{[a,c]} \land f \in R_{[c,b]}
        .\] 
    \end{theorem}
    То  $f \in R[a,b]$ и  $\int\limits_{a}^{b} f = \int\limits_{a}^{c } f + \int\limits_{c}^{b} f$ 
\end{enumerate}
\begin{corollary}
    \[
        f \in R_{[a,b]}
    .\] 
    Тогда $|\int\limits_{a}^{b}|  \le  \int\limits_{a}^{b}  |f|$
\end{corollary}
\begin{proof}
    \[
    -|f| \le  f \le |f|
    .\] 
    \[
    -\int\limits_{a}^{b}  |f|  \le  \int\limits_{a}^{b} f \le  \int\limits_{a}^{b} |f| 
    .\] 
    \[
    |\int\limits_{a}^{b} f | \le \int\limits_{a}^{b} |f| 
    .\] 
    При $a \le  b$
\end{proof}
В общем случае
\[
|\int\limits_{a}^{b} f| \le  |\int\limits_{a}^{b} |f| | 
.\] 
\[
|\int\limits_{a}^{b} f| = | -\int\limits_{b}^{a} f|  = |\int\limits_{b}^{a}  f|
.\] 
\section{Суммы Дарбу}
\begin{definition}
    Пусть $f : [a,b] \to \mathbb{R}$ , $\tau = \{x_1,x_2,\dots,x_{n}\}$ -- разбиение 
    \[
    d(f,\tau) = \sum_{i = 1}^{n} m_{i} \Delta x_{i}
    .\] 
    \[
        m_{i}= \inf_{x \in [x_{i - 1},x_{i}]} f(x)
    .\] 
    \[
    D(f,\tau) = \sum_{i = 1}^{n} M_{i} \Delta x_{i}
    .\] 
    \[
        M_{i} = \sum_{x \in [x_{i-1},x_{i}]} f(x)
    .\] 
\end{definition}
\subsection{Свойства}
\begin{enumerate}
    \item $d(f,\tau) \le  S(f,(\tau,\xi) \le D(f,\tau)$ \\
     $\forall i ~ m_{i} \le f(\xi_{i}) < M_{i}$
     \[
     m_{i} \Delta x_{i} < f(\xi_{i}) \Delta x_{i} \le  M_{i} \Delta x_{i}
     .\]
     \[
     \sum_{i = 1}^{n} m_{i} \Delta x_{i} \le  \sum f(\xi_{i}) \Delta x_{i} \le 
     \sum_{i = 1}^{n} M_{i} \Delta x_{i}
     .\] 
\item
      \[
     d(f,\tau) = \inf_{\xi} S(f,(\tau,\xi))
     .\] 
     \[
     D(f,\tau) = \sup_{\xi} S(f,(\tau,\xi))
     .\] 
     \begin{proof}
         \[
             \forall  \epsilon > 0 \exists \xi_{i} \in [x_{i -1},x_{i}] ~
             M_{i} - f(\xi_{i}) < \epsilon
         .\] 
         \[
         ( M_{i} - f(\xi_{i}) ) \Delta x_{i} < e \Delta x_{i}
         .\] 
         \[
         \sum_{i = 1}^{n} (M_{i} - f(\xi_{i}))\Delta x_{i} \le 
         .\] 
         \[
         D(f,\tau) - S(f,(\tau,\xi)) = \sum_{i = 1}^{n} (M_{i} - f(\xi_{i})) \delta x_{i}
         .\] 
     \end{proof}
 \item  
     \[
         f:[a,b] \to \mathbb{R}
     .\] 
     \[
         \tau,\tau' -- \text{Разбиения отрезка} [a,b]
     .\] 
     \[
     \tau' \supset \tau
     .\] 
     Тогда
     \[
     D(f,\tau') \le  D(f,\tau)
     .\] 
     \[
     d(f,\tau') \ge d(f,\tau)
     .\] 
     \begin{proof}
         Достаточно доказать только для случая $\tau' = \tau \cup \{c\}$
         Суммы дарбу меняются только в том месте, куда попадапет c.
         \[
         D(f,\tau') = \dots + (M_{i}^{*} (c - x_{i - 1}) + M_{i}^{*}(x_{i} - c)) + \dots
         .\] 
         \[
         M_{i}((c - x_{i - 1}) + (x_{i} - c)) = M_{i}(c - x_{i - 1}) + M_{i}(x_{i} - c) \ge  M_{i}^{*}+M_{i}^{**}
         .\] 
         \[
         M_{i} \ge  M_{i}^{*}
         .\] 
         \[
         M_{i} \ge  M_{i}^{**}
         .\] 
     \end{proof}
    \item
        \[
            \forall  \tau_1 \tau_2 ~\text{разбиений} ~ [a,b]
        .\] 
        \[
            \tau' = \tau_1 \cap \tau_2
        .\] 
        \[
        d(f,\tau_1) \le  d(f,\tau') \le D(f,\tau') \le D(f,\tau_{2})
        .\] 
\end{enumerate}
\subsection{Критерий интегрируемости в терминах Дарбу}
Пусть $f:[a,b] \to \mathbb{R}$
\[
    f \in R_{[a,b]} \iff \forall \epsilon > 0 ~ \exists \delta >0 \forall  \tau 
    \lambda(\tau) < \delta : D(f,\tau) - d(f,\tau) < \epsilon
.\] 
\begin{proof}
     \begin{enumerate}
         \item Пусть функция интегрируема $f \in R_{[a,b]}$ по определению
             $\exists I$ что $\forall  \varepsilon > 0 $ $|S(f,(\tau,\xi)) - I| < \varepsilon$
             Для всех достаточно мелих разбиений
             \[
              I - \varepsilon < S(f,(\tau, \xi  )) < I + \varepsilon
             .\] 
             \[
                 I + \varepsilon ~ \text{верхняя граница инт сум для разбиения} \tau
             .\] 
             Верхняя сумма дарбу меньше или равна этой херне, так как она инфинум верхних границ
             C нижней аналогично
             \[
             d(f,\tau) < S(f,(\tau, \xi  )) < D(f,\tau)
             .\] 
             \[
                 d(f,\tau) - D(f,\tau) < \varepsilon
             .\] 
        \item
            \[
            S(f,(\tau, \xi))
            .\] 
    \end{enumerate}
\end{proof}
\begin{theorem}
    \[
        f \in R_{[a,b]} , [c,d] \subset [a,b]
    .\] 
    Тогда $f \in R_{[c,d]}$
\end{theorem}
\begin{proof}
    Докажем, что $\forall  \varepsilon > 0$ $D(f|_{[c,d]},\tau^{*}) - d(f|_{[c,d]},\tau^{*}) < \varepsilon$
\end{proof}
\begin{theorem} 
    Если $f \in C_{[a,b]}$ , то $f \in R_{[a,b]}$
\end{theorem}
\begin{proof}
    Так как $f \in C_{[a,b]}$ f равномерно непрерывна 
    \[
        \forall  \varepsilon > 0 \exists  \delta > 0 \forall x,y \in [a,b] |x - y| < \delta
    .\] 
    \[
        |f(x) - f(y)| < \varepsilon
    .\] 
    \[
    D(f,\tau) - d(f,\tau) = \sum M_{i}\Delta x_{i} - \sum m \Delta x_{i} = 
    \sum (M_{i} - m_{i}) \Delta x_{i}
    .\]
\end{proof}
\begin{theorem}
    Пусть $f$ задана на отрезке  $[a,b]$ f монотонна, тогда  $f$ интегрируема
\end{theorem}
\section{Интеграл с переменным верхним пределом}
Пусть функция f задана на промежутке $D$, f интегрируема на любом промежутке лежащем внутри D. Такие функции называют локально интегрируемыми.
Зададим на D функцию F формулой выберем $\forall  a \in D$
\[
F(x) = \int\limits_{a}^{x} f(t) dt
.\] 
\begin{theorem}
    Пусть f задана на D, $a \in D$ тогда функция  $F(x) = \int\limits_{a}^{x} f 
    $ 
    Непрерывна
\end{theorem}
\begin{proof}
    \[
    \forall  x_0 \in D
    .\] 
    \[
    |F(x) - F(x_0)| = | \int\limits_{a}^{x} f - \int\limits_{a}^{x_0} f | =
    |\int\limits_{x_0}^{x} f|  \le |\int\limits_{x_0}^{x} |f| | \le |\int\limits_{x_0}^{x} | =  |M| |x - x_0|
    .\] 
    Так f интегрируема на промежутке с концами $x_0,x$, она ограничена на этом промежутке
    \[
    0 \le  |F(x) - F(x_0) | \le  M |x - x_0|
    .\] 
    По теорме о ментах $|F(x) - F(x_0)| \to 0 x \to x_0$
\end{proof}
\begin{theorem}(Барроу)
    \[
    f:D \to \mathbb{R}
    .\] 
    f локально интегрирумема
    \[
    F(x) = \int\limits_{a}^{x} f 
    .\] 
    Пусть $x_{0} \in D$ и пусть $f$ непрерывна в точке  $x_0$
    Тогда F дифференцируема в $x_0$ и $F'(x) = f(x_0)$ 
    \[
    |F(x) - F(x_0) - f(x_0) (x - x_0)|
    .\] 
    \[
    |\int\limits_{a}^{x} f - \int\limits_{a}^{x_0} f - \int\limits_{x_0}^{x} f(x_0) |
    .\] 
    \[
    |\int\limits_{x_0}^{x} f - \int\limits_{x_0}^{x} f(x_0) = |\int\limits_{x_0}^{x} (f - f(x_0)) \le  |\int\limits_{x_0}^{x} |f - f(x_0) ||
    .\] 
    Так как f непрерывна в $x_{0}$, то
    \[
    \forall  \varepsilon > 0 \exists  \delta >0  |t- x_0| < \delta \implies |f(t) - f(x_0) | < \varepsilon
    .\] 
\end{theorem}
\begin{corollary}
    $f \in C_{[a,b]}$. Тогда f имеет первобразную
    \[
    F'(x_0) = f(x_0)
    .\] 
\end{corollary}
\section{Обобщенная первообразная}
\begin{definition}
    \[
    f : D \to \mathbb{R}
    .\] 
    K -- конечное подмножество $D$\\
    Пусть  $F \in C_{[a,b]}$
    F обобщенная первообразная функции f, если для всех точек K
    \[
    F'(x) = f(x)
    .\] 
\end{definition}
\begin{theorem}
    Любые две обобщенные первообразные отличаются на константу
\end{theorem}
\section{Ряды}
Есть последовательность $a_1,a_2,\dots$
\[
a_1 + a_2 + \dots
.\] 
Вот эта фигня с плюсиками ряд.
\[
\sum_{n=1}^{\infty} a_{n}
.\] 
\[
S_1 = a_1
.\] 
\[
S_2 = a_1 + a_{2}
.\] 
\[
S_{n} = \sum_{i = 1}^{n} a_{i}
.\] 
\begin{definition}[Сумма ряда]
    \[
    \lim_{n \to \infty} S_{n}
    .\] 
    Сумма -- конечна ряд сходится иначе расходится.
\end{definition}
\subsection{Свойства}
\begin{enumerate}
    \item Если ряд сходится, то его последовательность стремится к нулю.
\end{enumerate}
\section{Несобственные интегралы}
\begin{definition}[Несобственный интеграл]
    \[
    \lim_{t \to + \infty} F(t) = \int\limits_{a}^{+\infty} f(x) dx
    .\] 
\end{definition}
Сходится если эта фигня конечная
\subsection{Примеры}
\begin{enumerate}
    \item 
        \[
        f(x) = \frac{1}{x}
        .\] 
        \[
        \int\limits_{1}^{+\infty} \frac{1}{x}dx 
        .\] 
        \[
        \lim_{t \to +\infty} \int\limits_{1}^{t} \frac{1}{x}dx 
        .\] 
        \[
            \lim_{t \to +\infty} \ln{t} = +\infty
        .\] 
    \item
        \[
        \int\limits_{1}^{+\infty}  \frac{1}{x^2}dx = \lim_{t \to +\infty} \int\limits_{1}^{t} \frac{1}{x^2} dx =
        \lim_{t \to +\infty} (-\frac{1}{t} + 1) = 1
        .\] 
\end{enumerate}
\section{Снова ряды}
\[
    \sum_{n=1}^{\infty} a_{n}
.\] 
\[
S_1 = a_1
.\] 
\[
S_2 = a_1 + a_2
.\] 
\[
S_{n} = \sum_{i = 1}^{n} a_{i}
.\] 
\begin{definition}
    Суммой ряда называется
     \[
    \lim_{n \to \infty} S_{n}
    .\] 
\end{definition}
Ряд сходится если сумма конечная
\subsection{Гармонический ряд}
\[
1 + \frac{1}{2} + \frac{1}{3} + \dots
.\] 
\subsection{Теоремочки}
\begin{theorem}
    \[
    \sum_{n=1}^{\infty} a_{n} = S
    .\] 
    \[
    \sum_{n=1}^{\infty} b_{n} = T
    .\] 
    Тогда
    \[
    \sum_{n=1}^{\infty} (a_{n} + b_{n}) = S + T
    .\] 
\end{theorem}
\begin{proof}
    Пусть  $U_{n}$ n-я частичная сумма ряда $\sum_{n=1}^{\infty} (a_{n} + b_{n})$
    \[
    U_{n} (a_1 + b_1) + \dots + (a_{n} +b_{n}) = (a_1 + a_2 + \dots + a_{n}) + (b_1 + b_2 + \dots + b_{n}) = S_{n} + T_{n}
    .\] 
    \[
    \lim_{n \to \infty}U_{n} = \lim_{n \to \infty} S_{n} + \lim_{n \to \infty} T_{n}
    .\] 
\end{proof}
\begin{theorem}
    Пусть 
    \[
        \sum_{n=1}^{\infty} a_{n} = S
    .\] 
    \[
        \alpha \in \mathbb{R}
    .\] 
    Тогда
    \[
    \sum_{n=1}^{\infty} \alpha a_{n} = \alpha S
    .\] 
\end{theorem}
\begin{proof}
    \[
    U_{n} = \alpha (a_1 + a_2 + \dots+ a_{n})
    .\] 
    \[
    \lim_{n \to \infty} U_{n} = \alpha \lim_{n \to \infty} S_{n} = \alpha S
    .\] 
\end{proof}
\subsection{Ряды с неотрицательными членами}
\begin{theorem}[Необходимое и достачное условие сходимсоть ряда, все члены которого неотрицательные]
    Ряд, члены вкоторого неотрицательные, сходится тогда и только тогда когда множество его частичных суммм ограниченно.
\end{theorem}
\begin{proof}
    \[
    a_1+a_2 +a_3 +\dots
    .\] 
    \[
        S_i < S_{i + 1}
    .\] 
    По т Вейештрасса $S_{n}$ имеет конечный предел ряд сходится
\end{proof}
\begin{theorem}[Первый признак сравнения]
    Даны ряды
    \[
    \sum_{n=1}^{\infty} a_{n}
    .\] 
    \[
        \sum_{n=1}^{\infty} b_{n}
    .\] 
    $\forall n ~ a_{n} \le  b_{n}$ 
    Тогда если больший ряд сходится, меньший ряд сходится (если меньший расходится, то больший расходится)
\end{theorem}
\begin{proof}
    Пусть  $\sum_{n=1}^{\infty} b_{n} = T$ 
    Последовательность $T_{n}$ ограничена , тогда частичные суммы первого ряда тоже ограничены.
\end{proof}
\begin{theorem}[Предельный признак сравнения]
    \[
    \sum_{n=1}^{\infty} a_{n}
    .\] 
    \[
    \sum_{n=1}^{\infty} b_{n}
    .\] 
    \[
    b_{n} > 0
    .\] 
    Пусть $\lim_{n \to \infty} \frac{a_{n}}{b_{n}} = C \neq 0$ тогда первый ряд сходится тогда и только тогда когда второй ряд сходится
\end{theorem}
\begin{proof}
    \[
    \frac{a_{n}}{b_{n}} \in (K;L)
    .\] 
    \[
    K * b_{n} < a_{n} < L * b_{n}
    .\] 
    Пусть $a_{n}$ cходится, тогда $\sum_{n=1}^{\infty}K b_{n}$ сходится,  а значит $\sum_{n=1}^{\infty} b_{n}$ 
    Пусть $\sum_{n=1}^{\infty} b_{n}$ тогда $\sum_{n=1}^{\infty} L b_{n}$ сходится, а значит $\sum_{n=1}^{\infty} a_{n}$
\end{proof}
\subsubsection{Пример}
\[
\sum_{n=1}^{\infty} \frac{1}{n^2 - 100n + 2}
.\] 
\[
\frac{n^2}{n^2  -100n + 2} \to 1
.\] 
\section{Еще признаки сравнения}
\subsection{Признак Коши}
\begin{theorem}
    Дан ряд $\sum_{n=1}^{\infty} a_{n} , a_{n} \ge  0 \forall n$ 
    Рассмотрим последовательность 
    \[
        c_{n} = \sqrt[n]{a_{n}} 
    .\] 
    Пусть $c_{n}$ имеет предел
    \[
    \lim_{n \to \infty} c_{n} = C
    .\] 
    Тогда
    \begin{enumerate}
        \item если $c < 1$ ряд сходится.
        \item  если  $c > 1$ ряд расходится.
        \item  $c = 1$ ничего не знаем, ряд может и сходиться и расходиться.
    \end{enumerate}
\end{theorem}
\begin{proof}
    \begin{enumerate}
        \item Пусть $c < 1$
            Выберем q ,  $c < q < 1$,  $c_{n} \to C$, то $\exists  n_0 \forall  n\ge n_0 c_{n} < q$
             \[
                 \sqrt[n]{a_{n}}  < q
             .\] 
             \[
             a_{n} < q^{n}
             .\] 
             Рассмотрим ряд $\sum_{n=1}^{\infty} q^{n}$. Этот ряд сходится, по признаку сравнения $\sum_{n=1}^{\infty} a_{n}$ сходится
            \item
                Пусть $c > 1$  $\exists n_0 \forall n\ge n_0 ~ a_{n} > 1$ 
            \item $\sum_{n=1}^{\infty} \frac{1}{n}, \sum_{n=1}^{\infty} \frac{1}{n^2} $
    \end{enumerate}
\end{proof}
\subsection{Признак Даламбера}
\begin{theorem}
    Дан ряд $\sum_{n=1}^{\infty} a_{n}$ 
    Рассмотрим последовательность
    \[
    d_{n} = \frac{a_{n + 1}}{a_{n}}
    .\] 
    Пусть $d_{n}$ имеет предел $\lim_{n \to \infty} d_{n} = d$ 
    \begin{enumerate}
        \item $d < 1$ ряд сходится
        \item  $d > 1$ ряд расходится
        \item  $d = 1$ не понятно
    \end{enumerate}
\end{theorem}
\begin{proof}
    \begin{enumerate}
        \item $d < q , 1$  $\exists  n_0  ~ \forall  n \ge  n_0$ $\frac{a_{n + 1}}{n} < q$ 
            \[
            \frac{a_{n_0 + 1}}{a_{n_0}} < q\dots \frac{a_{n_0 + k}}{a_{n_0 + k -1}} < q
            .\] 
            \[
                a_{n_{0 + k}} < a_{n_0} q^{k}
            .\] 
            Рассмотрим ряд $\sum_{k=1}^{\infty} a_{n_0 + k}$ вынули из исходного первые $n_0$ членов и ряд $\sum_{n=1}^{\infty} a_{n_0}q^{k}$ 
            По признаку сравнения ряд сходится, а значит исходный сходится
        \item
            $d > 1$  $\frac{a_{n + 1}}{a_{n}} > 1$ $a_{n + 1} > a_{n}$ ($n \ge  n_0$)
            Последовательность начиная с $n_0$ строго возрастает, не стримится к нулю, ряд раcходится
    \end{enumerate}
\end{proof}
\subsection{Примеры}
\begin{enumerate}
    \item $\sum_{n=1}^{\infty} \frac{n^3}{3^{n}}$ 
        \begin{enumerate}
            \item Коши
                \[
                    \sqrt[n]{\frac{n^{3}}{3^{n}}}  = \frac{\sqrt[3]{n^{3}} }{3} \to \frac{1}{3}
                .\] 
            \item $\frac{(n + 1)^{3}}{3^{n+1}} * \frac{3^{n}}{n^{3}} = (\frac{n + 1}{3}) * \frac{1}{3} \to \frac{1}{3}$
        \end{enumerate}
    \item $\sum_{n=1}^{\infty} \frac{n^{10}}{(n + 1)!}$ 
        \[
        \frac{(n + 1)^{10}}{(n + 2)!} * \frac{(n + 1)!}{n^{10}} =
        (\frac{n + 1}{n})^{10} * \frac{1}{n + 2} \to 0 
        .\] 
\end{enumerate}
\subsection{Интегральный признак}
\begin{theorem}
    Пусть $f$ задана на  $[1,+\infty]$ Пусть f интегрируема на каждом промежутке $[a,b] \subset [1;+\infty]$ f убывает на всем промежутке. Тогда ряд $\sum_{n=1}^{\infty} f(n)$ сходится $\iff$  $\int\limits_{1}^{+\infty} f(x) dx$   сходится. 
\end{theorem}
\begin{proof}
    Рассмотрим частичную сумму 
    \[
    S_{n} = f(1) + f(2) + \dots + f(n - 1) + f(n)
    .\] 
    \[
    S_{n} > \int\limits_{1}^{n + 1} f(x)  dx
    .\] 
    \[
    f(2) + \dots + f(n) < \int\limits_{1}^{n} f(x)  dx
    .\] 
    \begin{enumerate}
        \item Пусть $\int\limits_{1}^{+\infty} f(x)  dx$ Сходится
            \[
            S_{n} < f(1) + \int\limits_{1}^{n} f(x) dx \le f(1) + \int\limits_{1}^{+\infty} f(x)   dx
            .\] 
        \item Интеграл расходится есть бесконечный предел
            Имеем такое неравенство
            \[
            S_{n} > \int\limits_{1}^{n+ 1} f(x) dx
            .\] 
            \[
            \lim_{n \to \infty} \int\limits_{1}^{n+1} f = +\infty 
            .\] 
            Ряд расходится
    \end{enumerate}
\end{proof}
\subsubsection{Пример}
\[
    \sum_{n=2}^{\infty} \frac{1}{n \ln{n}}
.\] 
\[
    \int \frac{1}{x \ln{x}} dx = \int \frac{1}{\ln{x}} d \ln{x} = \ln{\ln{x}}
.\] 
\[
    \lim_{x \to \infty} \ln{\ln{x}} - \ln{\ln{2}} = \infty
.\] 
\section{Абсолютно сходящиеся ряды}
Дан ряд $\sum_{n=1}^{\infty} a_{n}$ 
\begin{definition}
    Если сходится ряд $\sum_{n=1}^{\infty} |a_{n}|$, то $\sum_{n=1}^{\infty} a_{n}$ 
    абсолютно сходится
\end{definition}
\begin{theorem}
    Если ряд сходится абсолютно, то он сходится 
\end{theorem}
\begin{proof}
    Если отрицательных чисел конечно, то ряды почти одинаковые (отбрасывание конечного числа не влияет на сходимость)

    Пусть в $\sum_{n=1}^{\infty} a_{n}$ бесконечное число неотрицательных членов и бесконечное число отрицательных членов

    Построим 2 ряда. $(*)$ получили из исходного заменой всех отрицательных на 0. $(**)$ получили из исходного ряда заменой всех неотрицательных членов на 0, а отрицательных членов на их модули. Сумма новых рядов это исходный из модулей, разность -- исходный
    
    Пусть ряд из модулей сходится, тогда  $(**)$ сходится,  $(*)$ сходится, тогда  $(*) - (**)$ сходится
\end{proof}
\begin{definition}
    Если ряд сходится, а ряд из модулей расходится, то такой ряд называется условно сходящимся
\end{definition}
\end{document}
