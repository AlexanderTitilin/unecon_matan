\documentclass{scrartcl}
\usepackage[utf8]{inputenc}
\usepackage[T2A]{fontenc}
\usepackage[russian]{babel}
\usepackage{amssymb}
\usepackage{amsmath}
\usepackage{amsthm}
\usepackage{listings}
\newtheorem{theorem}{Теорема}
\newtheorem{definition}{Определение}
\newtheorem{corollary}{Следствие}[theorem]
\newtheorem{lemma}[theorem]{Лемма}
\title{Лекции по математическому анализу}
\author{Титилин Александр}
\date{}
\begin{document}
    \maketitle
    \section{Равномерная непрерывность.}
    \[
    \forall x \in D \forall  \varepsilon > 0 \exists  \delta \forall y |x - y| < \delta\implies |f(x) -f(y)| < \varepsilon
    .\] 
    Определение непрерывной функции.
    \[
    \forall  \varepsilon > 0 \exists  \delta >0 \forall x \forall y |x- y| < \delta \implies |f(x) - f(y)| < \varepsilon
    .\] 
    Определение равномерно непрерывной функции.
    \subsection{Примеры}
    \begin{enumerate}
        \item $f(x) =  C$
            Берем любую дельту
             \[
                 |f(x) - f(y)| = 0 < \varepsilon
            .\] 
        \item $f(x) = ax + b$
            \[
            |f(x) - f(y)| = |ax + b - ay -b| =|a(x - y)| = |a||x - y|
            .\] 
            \begin{enumerate}
                \item $a = 0 \implies$ прошлый пункт
                \item $a \neq 0$ 
                    \[
                    \delta = \frac{\varepsilon}{|a|}
                    .\] 
            \end{enumerate}
        \item $f(x) =  x^2 D = [a,b] a\neq b$
            \[
            \varepsilon > 0
            .\] 
            \[
            |x^2 - y^2| = |(x - y)||(x + y)| \le  (|x| + |y|)|x - y| \le  2C|x - y|
            .\] 
            \[
            C = \max(|a| , |b|)
            .\] 
            \[
            \delta = \frac{\varepsilon}{2C}
            .\] 
        \item $f(x) = \frac{x^2}{x^2 + 1}$
            \[
            | \frac{x^2}{x^2 + 1} - \frac{y^2}{y^2 + 1} | =
            |\frac{x^2y^2 + x^2 - x^2y^2 - y^2}{(x^2 + 1)(y^2 + 1)}| =
            |\frac{x^2 - y^2}{(x^2 + 1)(y^2 + 1)}| =
            |\frac{(x + y)(x - y)}{(x^2 + 1)(y^2 + 1)}|
            .\] 
    \end{enumerate}
    \begin{definition}
        Пусть $f: D \to \mathbb{R}$ говорят, что функция $f$ удовлетворяет условию Липшица
     если 
      \[
     \exists  C \ge  0 \forall  x, y \in D : |f(x) - f(y)| < C|x - y|
     .\] 
    \end{definition}
    \begin{theorem}
        Если функция удовлетворяет условию Липщица на промежутке D, то она равномерно
        непрерывна на промежутке D.
    \end{theorem}
    \begin{theorem}
        Пусть функция задана на промежутке, функция удовлетворятет условию Гёльде, 
        если
        \[
        \exists  C \ge  0 \exists  \alpha > 0 \forall  x, y \in D : |f(x)-f(y)|\le  C|x - y|^{\alpha}
        .\] 
    \end{theorem}
    \begin{theorem}
        Если функция удовлетворяет условию Гёльдера, то она равномерно непрерывна.
    \end{theorem}
    \begin{proof}
        \[
        C > 0 ~  |x - y| < (\frac{\varepsilon}{C})^{\frac{1}{\alpha}}
        .\] 
        \[
        \delta = (\frac{\varepsilon}{C})^{\frac{1}{\alpha}}
        .\] 
    \end{proof}
     \subsection{Пример.}
     \[
         D = (0,+\infty) , f(x) = \sqrt{x} 
     .\] 
     \[
     |f(x) - f(y)| = |\sqrt{x}  - \sqrt{y} |
     .\] 
     \begin{theorem}
         Функция f равномерно непрерывна $\iff$ $\forall  (x_{n}), (y_{n}), x_{n}, y_{n} \in D 
         \implies f(x_{n}) - f(y_{n}) \to 0$
     \end{theorem}
     \begin{proof}
         \begin{enumerate}
             \item 
                 Возьмем $\forall (x_{n}) (y_{n}) x_{n} - y_{n} \to 0$.
                 Нужно доказать $\forall \varepsilon >0 \exists  n_0 \forall  n\ge n_0
                 |f(x_{n}) - f(y_{n})| < \varepsilon$
                 Дальше смотрим определение равномерной непрырывности и все понятно.
            \item
                Теперь обратно доказываем. От противного
                \[
                \exists  \varepsilon > 0 \forall \delta > 0 |x - y| < \delta \land |f(x) - f(y)| > \varepsilon
                .\] 
                Берем все такие последовательность $(x_{n}) (y_{n}) x_{n} - y_{n} < \frac{1}{n}$ 
                \[
                |x_{n} - y_{n}| \to 0 \implies f(x_{n}) - f(y_{n}) \to 0
                .\] 
                По т. о ментах.
         \end{enumerate}
     \end{proof}
     \subsection{Пример функции, которая не является равномерно непрерывной}
     \[
     D = (0,+\infty) f(x) = \frac{1}{x}
     .\] 
     \[
     x_{n} = \frac{1}{n}
     .\] 
     \[
     y_{n} = \frac{1}{n + 1}
     .\] 
     \[
     x_{n} - y_{n}= \frac{1}{n(n + 1)} \to 0
     .\] 
     \[
     f(x_{n}) = n
     .\] 
     \[
     f(y_{n}) = n + 1
     .\] 
     \[
     f(x_{n}) - f(y_{n}) = -1 \not \to 0
     .\] 
\subsection{Еще такой пример}
\[
    f(x) = \ln{x} ~ D = (0;+\infty)
.\] 
\[
x_{n} = \frac{1}{n}
.\] 
\[
y_{n} = \frac{1}{n^2}
.\] 
\[
x_{n} - y_{n} \to 0
.\] 
\[
    f(x_{n}) - f(y_{n}) = \ln{x_{n}} - \ln{y_{n}} = \ln{\frac{x_{n}}{y_{n}}} = \ln{0} \to +\infty
.\] 
\subsection{}
\[
f(x) = x^2, D =\mathbb{R}
.\] 
\[
x_{n} =  n + \frac{1}{n}
.\] 
\[
y_{n} = n
.\] 
\[
x_{n} - y_{n} \to 0
.\] 
\[
f(x_{n}) = n^2 + 2 + \frac{1}{n^2}
.\] 
\[
f(y_{n}) =n ^2
.\] 
\[
f(x_{n}) - f(y_{2}) = 2 + \frac{1}{n} \to 2
.\] 
\subsection{}
\[
    f(x) = \tg{x} D = (-\frac{\pi}{2};\frac{\pi}{2})
.\] 
\[
x_{n} = \frac{\pi}{2} - \frac{1}{n}
.\] 
\[
y_{n} = \frac{\pi}{2} - \frac{1}{2n}
.\] 
\[
    \tg{x_{n}} - \tg{y_{n}} \not \to 0
.\] 
\subsection{Пример}
\[
    D = [0,\frac{1}{2}]
.\] 
\[
f(x) = 
\begin{cases}
    \frac{1}{\ln{x}}, \text{если } x \neq 0\\
    0, \text{если } x = 0
\end{cases}
.\] 
Докажем, что не удовлетворяет условию Гельдера.\\
Пусть $\exists  C >0 \alpha >0 $
\[
    \forall  x, y \in [0; \frac{1}{2}] ~ |f(x) - f(y)| \le  C |x - y|^{\alpha}
.\] 
\section{Теорема Кантора}
\begin{theorem}[Кантора]
    Пусть функция f задана на замкнутом и ограниченном промежутке и непрерывна.Тогда $f$ равномерна непрерывна.
\end{theorem}
\begin{proof}
    Пусть $f:[a,b] \to \mathbb{R}$ непрерывна на $[a,b]$, но не является равномерно непрерывной.
    Тогда $\exists  \varepsilon \forall  \delta >0$
    Найдутся такие $x,y \in[a,b]$, такие что  $|x - y| < \delta \land |f(x) - f(y)| \ge \varepsilon$.
    В частности $\forall  n \in \mathbb{N}$ найдутся такие $x_{n},y_{n}$,
    что $|x_{n} - y_{n}| < \frac{1}{n} \land |f(x_{n}) - f(y_{n}) \ge  \varepsilon$
    Рассматриваем последовательность $(x_{n})$ Выбрали из нее сходящую подпоследовательность $(x_{n_{k}}) \to c c \in [a;b]$
    Рассмотрим $y_{n_{k}}$ 
    \[
    y_{n_{k}} = (y_{n_{k}} - x_{n_{k}}) + x_{n_{k}}
    .\] 
    По ментам
    \[
    0 \le  |y_{n_{k}} - x_{n_{k}}| <\frac{1}{n} \to 0
    .\] 
    f непрерывна в точке с $x_{n_{k}} \to c \implies f(x_{n_{k}}) \to f(c)$ 
    \[
    y_{n_{k}} \to c \implies  f(y_{n_{k}}) \to f(c)
    .\] 
    \[
    f(x_{n_{k}}) - f(y_{n_{k}}) \to 0
    .\] 
    Противоречие так как $|f(x_{n})  - f(y_{n})| \ge  \varepsilon$
\end{proof}
\begin{theorem}
    \[
    d : (a,b) \to \mathbb{R}
    .\] 
    Пусть f непрерывна  f равномерно непрерывна $\iff$
    Существуют конечные пределы   $\lim_{x \to a^{+}} f(x) ,
    \lim_{x \to b^{-}}f(x) $
\end{theorem}
\begin{proof}
    В обратную сторону. Пусть 
    \[
    \lim_{x \to a^{+}}  f(x) = A
    .\] 
    \[
    \lim_{x \to b^{-}}  f(x) = B
    .\] 
    \[
        \overline{f} [a,b] \to \mathbb{R}
    .\] 
    \[
    \overline{f(x)} =
    \begin{cases}
        f(x), x \in (a,b)\\
        A , x = a\\
        B, x = b\\
    \end{cases}
    .\] 
    $\overline{f}$ непрерывна в $x \in (a,b)$ в  $a,b$
\end{proof}
\subsection{Примеры}
$f(x) = \sqrt{x} ~ D = (0,1) $ Так же она равномерно непрерывна на $[a,b]$
 \[
     \lim_{x \to 0^{+}} f(x) = 0
 .\] 
 \[
 \lim_{x \to 1^{-}} f(x) = 1
 .\] 
 \[
     f(x) = \tg{x} ~ D = (-\frac{\pi}{2}; \frac{\pi}{2})
 .\] 
 \[
 \lim_{x \to \frac{\pi}{2}^{+} }f(x)  = +\infty
 .\] 
 f не является равномерной непрерывной\\
 \[
     f(x) = \sqrt{1 - x^2}  ~D = [-1;1]
 .\] 
 По т Кантора равномерно непрерывна.
\section{}
\end{document}
